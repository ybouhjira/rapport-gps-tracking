\documentclass[a4paper]{article}

\usepackage{graphicx}
\usepackage{float}

\begin{document}
\section{Introduction}
Le but de ce project et de réaliser un application de tracking GPS, qui 
permet à un ensemble d'utilisateur de partager leur trajets en temps réel.

Afin de réliser cette application nous avons utiliser les outils suivant :

\section{Outils utilisés}
\subsection{Node.js}

\begin{figure}[H]
  \begin{center}
  \includegraphics[scale=0.5]{nodejs.png}
  \end{center}
\end{figure}

Node.js est une plateforme logicielle libre et événementielle en 
JavaScript orientée vers les applications réseau.

Elle utilise la machine virtuelle V8 et implémente sous 
licence MIT les spécifications CommonJS. 

Node.js contient une bibliothèque de 
serveur HTTP intégrée, ce qui rend possible de faire tourner un serveur 
web sans avoir besoin d'un logiciel externe comme Apache ou Lighttpd, 
et permettant de mieux contrôler la façon dont le serveur web fonctionne.

\end{document}

